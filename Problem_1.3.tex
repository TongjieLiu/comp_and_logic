% A problem solution for Computability and Logic, 5th edition
%
% Copyright (C) 2022 Tongjie Liu <tongjieandliu@gmail.com>.
% 
% This program is free software: you can redistribute it and/or modify
% it under the terms of the GNU General Public License as published by
% the Free Software Foundation, either version 3 of the License, or
% (at your option) any later version.
% 
% This program is distributed in the hope that it will be useful,
% but WITHOUT ANY WARRANTY; without even the implied warranty of
% MERCHANTABILITY or FITNESS FOR A PARTICULAR PURPOSE.  See the
% GNU General Public License for more details.
% 
% You should have received a copy of the GNU General Public License
% along with this program.  If not, see <https://www.gnu.org/licenses/>.

\documentclass{article}
\usepackage{amsmath}

\title{Problem 1.3}
\author{Tongjie Liu}


\begin{document}
\maketitle


\section{Part A}
The identity function $f(a) = a, \forall a \in \mathbf{A}$ is a
correspondence between sets $\mathbf{A}$ and $\mathbf{B}$.


\section{Part B}
If $\mathbf{A}$ is equinumerous with $\mathbf{B}$, then there must
be a correspondence $f$ between $\mathbf{A}$ and $\mathbf{B}$. Function
$f^{-1}$ is a correspondence between $\mathbf{B}$ and $\mathbf{A}$.


\section{Part C}
If $\mathbf{A}$ is equinumerous with $\mathbf{B}$ and $\mathbf{B}$ is
equinumerous with $\mathbf{C}$, then there must be a correspondence
$f$ between $\mathbf{A}$ and $\mathbf{B}$, and a $g$ between $\mathbf{B}$
and $\mathbf{C}$. Function $h = gf$ is a correspondence between $\mathbf{A}$
and $\mathbf{C}$.
\end{document}
