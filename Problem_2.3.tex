% A problem solution for Computability and Logic, 5th edition
%
% Copyright (C) 2022 Tongjie Liu <tongjieandliu@gmail.com>.
% 
% This program is free software: you can redistribute it and/or modify
% it under the terms of the GNU General Public License as published by
% the Free Software Foundation, either version 3 of the License, or
% (at your option) any later version.
% 
% This program is distributed in the hope that it will be useful,
% but WITHOUT ANY WARRANTY; without even the implied warranty of
% MERCHANTABILITY or FITNESS FOR A PARTICULAR PURPOSE.  See the
% GNU General Public License for more details.
% 
% You should have received a copy of the GNU General Public License
% along with this program.  If not, see <https://www.gnu.org/licenses/>.

\documentclass{article}
\usepackage{amssymb}
\usepackage{amsmath}
\usepackage{amsthm}
\usepackage{IEEEtrantools}

\title{Problem 2.3}
\author{Tongjie Liu}


\begin{document}
\maketitle


\begin{proof}
	Our goal is to find a correspondence between the set of points on the
semicircle and the set of real numbers.

	If $\xi$ is a real number and $\xi \in (-1, 1)$, then $\xi$ has a
decimal expansion $\pm 0 . \xi_1 \xi_2 \xi_3 \ldots$ where each $\xi_i$ is one
of the cyphers 0-9. Some numbers have two decimal expansions, choose the one
with the 0s rather the one with the 9s. We define a function $g$ of $\xi$ by
assosiate each $\xi$ with an integer $\pm \xi_n \xi_{n - 1} \ldots \xi_2
\xi_1$, where $\xi_n$ is the least significant digit of $\xi$.

	Each point $(x, y)$ with $x \in (0, 1)$ and $y \in (0, 0.5)$ on the
semicircle could be associated to a pair $(a, b)$ with $a \in \mathbb{Z}$
and $b \in \mathbb{N}$ by a function $f$ defined by the following rules:
	\begin{IEEEeqnarray}{l}
		a = g(2(x - 0.5)), \nonumber \\
		b = g(2y). \nonumber
	\end{IEEEeqnarray}

	The correspondence we need is a function which associate each point
$(x, y)$ on the semicircle with a real number represented in decimal notation
as ``$a.b$'', where $a$ and $b$ are determined by pair $(a, b) = f(x, y)$.
\end{proof}
\end{document}
