% A problem solution for Computability and Logic, 5th edition
%
% Copyright (C) 2022 Tongjie Liu <tongjieandliu@gmail.com>.
% 
% This program is free software: you can redistribute it and/or modify
% it under the terms of the GNU General Public License as published by
% the Free Software Foundation, either version 3 of the License, or
% (at your option) any later version.
% 
% This program is distributed in the hope that it will be useful,
% but WITHOUT ANY WARRANTY; without even the implied warranty of
% MERCHANTABILITY or FITNESS FOR A PARTICULAR PURPOSE.  See the
% GNU General Public License for more details.
% 
% You should have received a copy of the GNU General Public License
% along with this program.  If not, see <https://www.gnu.org/licenses/>.

\documentclass{article}
\usepackage{amsmath}
\usepackage{amsthm}

\title{Problem 1.1}
\author{Tongjie Liu}


\begin{document}
\maketitle


\section{$f^{-1}$ is total if and only if $f$ is onto}
\begin{proof}
	In a more concise way, our goal is to prove:
	\begin{equation*}
		\forall b \in \mathbf{B}, \exists f^{-1}(b)
		\qquad \Leftrightarrow \qquad
		\forall b' \in \mathbf{B}, \exists a' \text{ such that }
		f(a') = b'
	\end{equation*}
	Since both $b$ and $b'$ represent any member of set $\mathbf{B}$,
	we could always choose a $b$ such that $b = b'$, and then substitute
	$b'$ with $b$:
	\begin{equation*}
		\forall b \in \mathbf{B}, \exists f^{-1}(b)
		\qquad \Leftrightarrow \qquad
		\forall b \in \mathbf{B}, \exists a \text{ such that }
		f(a) = b
	\end{equation*}
	Since $f^{-1}$ is the inverse function of $f$, $f^{-1}(b) = a$ if
	$f(a) = b$. In other words, both sides mean the same thing.
\end{proof}


\section{$f^{-1}$ is onto if and only if $f$ is total}
\begin{proof}
	In a more concise way, our goal is to prove:
	\begin{equation*}
		\forall a \in \mathbf{A}, \exists b \text{ such that }
		f^{-1}(b) = a
		\qquad \Leftrightarrow \qquad
		\forall a' \in \mathbf{A}, \exists f(a')
	\end{equation*}
	Since both $a$ and $a'$ represent any member of set $\mathbf{A}$,
	we could always choose a $a$ such that $a = a'$, and then substitute
	$a'$ with $a$:
	\begin{equation*}
		\forall a \in \mathbf{A}, \exists b \text{ such that }
		f^{-1}(b) = a
		\qquad \Leftrightarrow \qquad
		\forall a \in \mathbf{A}, \exists f(a)
	\end{equation*}
	Since $f^{-1}$ is the inverse function of $f$, $f(a) = b$ if
	$f^{-1}(b) = a$. In other words, both sides mean the same thing.
\end{proof}
\end{document}
