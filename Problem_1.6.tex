% A problem solution for Computability and Logic, 5th edition
%
% Copyright (C) 2022 Tongjie Liu <tongjieandliu@gmail.com>.
% 
% This program is free software: you can redistribute it and/or modify
% it under the terms of the GNU General Public License as published by
% the Free Software Foundation, either version 3 of the License, or
% (at your option) any later version.
% 
% This program is distributed in the hope that it will be useful,
% but WITHOUT ANY WARRANTY; without even the implied warranty of
% MERCHANTABILITY or FITNESS FOR A PARTICULAR PURPOSE.  See the
% GNU General Public License for more details.
% 
% You should have received a copy of the GNU General Public License
% along with this program.  If not, see <https://www.gnu.org/licenses/>.

\documentclass{article}
\usepackage{amsmath}
\usepackage{amsthm}

\title{Problem 1.6}
\author{Tongjie Liu}


\begin{document}
\maketitle


\begin{proof}
	According to Section 1.2 \textit{Enumerable Sets} of the book,
the set of finite sets of positive integers is an enumerable set, and
could be enumerated by a function $f$.

	Since a set $\mathbf{A}$ is an enumerable set, there must be a
function $g$ that could enumerate it. We could substitute each positive
integer $n$ in the range of function $f$ with a member $g(n)$ of set
$\mathbf{A}$. Then the new function $f'$ enumerates the set of all finite
subsets of the enumerable set $\mathbf{A}$.
\end{proof}
\end{document}
