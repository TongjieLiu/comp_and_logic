% A problem solution for Computability and Logic, 5th edition
%
% Copyright (C) 2022 Tongjie Liu <tongjieandliu@gmail.com>.
% 
% This program is free software: you can redistribute it and/or modify
% it under the terms of the GNU General Public License as published by
% the Free Software Foundation, either version 3 of the License, or
% (at your option) any later version.
% 
% This program is distributed in the hope that it will be useful,
% but WITHOUT ANY WARRANTY; without even the implied warranty of
% MERCHANTABILITY or FITNESS FOR A PARTICULAR PURPOSE.  See the
% GNU General Public License for more details.
% 
% You should have received a copy of the GNU General Public License
% along with this program.  If not, see <https://www.gnu.org/licenses/>.

\documentclass{article}
\usepackage{amsmath}
\usepackage{amsthm}

\title{Problem 1.2}
\author{Tongjie Liu}


\begin{document}
\maketitle


\section{Part A}
\begin{proof}
	The fact that $f$ and $g$ are both total could be written as:
	\begin{align*}
		\forall a \in \mathbf{A}, \exists f(a) \\
		\forall b \in \mathbf{B}, \exists g(b)
	\end{align*}
	Since $f(a) \in \mathbf{B}, \forall a \in \mathbf{A}$, we now have:
	\begin{equation*}
		\forall a \in \mathbf{A}, \exists g(f(a)) \qedhere
	\end{equation*}
\end{proof}


\section{Part B}
\begin{proof}
	The fact that $f$ and $g$ are both onto could be written as:
	\begin{align*}
		\forall b \in \mathbf{B}, \exists a \  
			such \  that \  f(a) = b \\
		\forall c \in \mathbf{C}, \exists b \ 
			such \  that \  g(b) = c
	\end{align*}
	This means that we now have:
	\begin{equation*}
		\forall c \in \mathbf{C}, \exists a \ 
			such \  that \  g(f(a)) = c \qedhere
	\end{equation*}
\end{proof}


\section{Part C}
\begin{proof}
	The fact that $f$ and $g$ are both one-to-one could be written as:
	\begin{align*}
		\forall b \in \mathbf{B}, a \  is \  unique \ 
			if \  \exists a \in \mathbf{A} \ 
			such \  that \  f(a) = b \\
		\forall c \in \mathbf{C}, b \  is \  unique \ 
			if \  \exists b \in \mathbf{B} \ 
			such \  that \  g(b) = c
	\end{align*}
	This means that we now have:
	\begin{equation*}
		\forall c \in \mathbf{C}, a \  is \  unique \ 
			if \  \exists a \in \mathbf{A} \ 
			such \  that \  g(f(a)) = c \qedhere
	\end{equation*}
\end{proof}
\end{document}
