% A problem solution for Computability and Logic, 5th edition
%
% Copyright (C) 2022 Tongjie Liu <tongjieandliu@gmail.com>.
% 
% This program is free software: you can redistribute it and/or modify
% it under the terms of the GNU General Public License as published by
% the Free Software Foundation, either version 3 of the License, or
% (at your option) any later version.
% 
% This program is distributed in the hope that it will be useful,
% but WITHOUT ANY WARRANTY; without even the implied warranty of
% MERCHANTABILITY or FITNESS FOR A PARTICULAR PURPOSE.  See the
% GNU General Public License for more details.
% 
% You should have received a copy of the GNU General Public License
% along with this program.  If not, see <https://www.gnu.org/licenses/>.

\documentclass{article}
\usepackage{amsmath}
\usepackage{amsthm}

\title{Problem 2.6}
\author{Tongjie Liu}


\begin{document}
\maketitle


\section{Part A}
\begin{proof}
	It's possible to define a function $f$ that associate each finite
sequence of rational numbers $n, c_d, c_{d - 1}, \ldots, c_1, c_0$ to
an algebraic number $x$, where $n$ is the rank of $x$ in all solutions of
equation $c_d x^d + c_{d - 1} x^{d - 1} + \ldots + c_1 x + c_0 = 0$
arranged in ascending order.

	When writing as fractions, a finite sequence of rational numbers
is described by a finite string of symbol ``0'', ``1'', ``2'', ``3'',
``4'', ``5'', ``6'', ``7'', ``8'', ``9'', `` '', ``/'', and ``,''.
\end{proof}


\section{Part B}
\begin{proof}
	According to Section 1.2 \textit{Enumerable Sets} of the book,
the set of rational numbers is enumerable. Since it is also infinite,
it is equinumerous with the set of positive integers. Furthermore, the
set of finite sequences of rational numbers is equinumerous with the
set of finite sequences of positive integers, which is enumerable according
to the same section of the book. Therefore, the set of finite sequences
of rational number is enumerable.

	Let us suppose that there is no transcendental number exist. In
other words, all real numbers are algebraic. As a consequence, the range
of function $f$ is the set of real numbers.

	According to the definition of algebraic number, $f$ is both onto
and total. For this reason, an enumeration of the set of finite sequences
of rational number could generates through function $f$ an enumeration of
the set of real numbers. But the set of real numbers is not enumerable,
which is Corollary 2.2.
\end{proof}
\end{document}
