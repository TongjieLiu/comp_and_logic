% A problem solution for Computability and Logic, 5th edition
%
% Copyright (C) 2022 Tongjie Liu <tongjieandliu@gmail.com>.
% 
% This program is free software: you can redistribute it and/or modify
% it under the terms of the GNU General Public License as published by
% the Free Software Foundation, either version 3 of the License, or
% (at your option) any later version.
% 
% This program is distributed in the hope that it will be useful,
% but WITHOUT ANY WARRANTY; without even the implied warranty of
% MERCHANTABILITY or FITNESS FOR A PARTICULAR PURPOSE.  See the
% GNU General Public License for more details.
% 
% You should have received a copy of the GNU General Public License
% along with this program.  If not, see <https://www.gnu.org/licenses/>.

\documentclass{article}
\usepackage{amsmath}
\usepackage{amsthm}

\title{Problem 1.5}
\author{Tongjie Liu}


\begin{document}
\maketitle


\begin{proof}
	According to Section 1.2 \textit{Enumerable Sets} of the book,
the set of rational numbers is an enumerable set. Since set (a) is a
subset of the set of rational numbers, it's an enumerable set itself.

	According to the same section, the set of finite sets of positive
integers is also an enumerable set. For this reason, there must be a
function $f$ that could enumerate this set. From the definition of cofinite,
we know that there is a finite set of positive integers for each cofinite
set of positive integers. If we substitue each finite set of positive integers
with its counterpart cofinite set of positive integers, then we'll have a
function $f'$ that could enumerate the set of cofinite sets of positive
integers. In other words, the set of cofinite sets of positive integers
is an enumerable set. Set (b) is an enumerable too, since it is the union
of two enumerable sets, namely the set of finite sets of positive integers
and the set of cofinite sets of positive integers.

	Sets (a) and (b) are enumerable and not finite, hence they are both
equinumerous with the set of all positive integers. According to Problem 1.3,
they are equinumerous. \qedhere
\end{proof}
\end{document}
