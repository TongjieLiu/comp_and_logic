% A problem solution for Computability and Logic, 5th edition
%
% Copyright (C) 2022 Tongjie Liu <tongjieandliu@gmail.com>.
% 
% This program is free software: you can redistribute it and/or modify
% it under the terms of the GNU General Public License as published by
% the Free Software Foundation, either version 3 of the License, or
% (at your option) any later version.
% 
% This program is distributed in the hope that it will be useful,
% but WITHOUT ANY WARRANTY; without even the implied warranty of
% MERCHANTABILITY or FITNESS FOR A PARTICULAR PURPOSE.  See the
% GNU General Public License for more details.
% 
% You should have received a copy of the GNU General Public License
% along with this program.  If not, see <https://www.gnu.org/licenses/>.

\documentclass{article}
\usepackage{amsmath}
\usepackage{amsthm}

\title{Problem 2.1}
\author{Tongjie Liu}


\theoremstyle{definition} \newtheorem*{def-delta-l}{$\mathbf{\Delta(L)}$}


\begin{document}
\maketitle


\begin{proof}
	If set $\mathbf{A}$ is an infinite enumerable set, then there is
a function $f$ to enumerate it.

	Let's suppose there is an enumeration $L$ of the set of all subsets
of set $\mathbf{A}$, then we could always define a speical set named
$\mathbf{\Delta(L)}$ as follows:
	\begin{def-delta-l}
		For each positive integer $n$, $f(n)$ is in set $\mathbf{
	\Delta(L)}$ if and only if $f(n)$ is not in set $\mathbf{S_n}$ which
	is the $n$th entry of enumeration $L$.
	\end{def-delta-l}

	Since the set of all subsets of set $\mathbf{A}$ is enumerable, then
set $\mathbf{\Delta(L)} \subseteq \mathbf{A}$ is in enumeration $L$. In other
words, for some positive integer $m$, $\mathbf{\Delta(L)} = \mathbf{S_m}$,
which is the $m$th entry of enumeration $L$.

	But according to the definition of set $\mathbf{\Delta(L)}$, $f(m) \in
\mathbf{S_m}$ is not in it. In other words, we've found a self-contradiction
which is $f(m) \in \mathbf{\Delta(L)}$ and $f(m) \notin \mathbf{\Delta(L)}$.
\end{proof}
\end{document}
